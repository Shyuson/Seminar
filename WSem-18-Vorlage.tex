\documentclass[twosided,a4,10pt]{article}
\usepackage[utf8]{inputenc}
\usepackage{amsmath}
\usepackage{amsfonts}
\usepackage{amssymb}
\usepackage{textcomp}
\usepackage{german}
\usepackage{graphicx}
\usepackage[usenames,dvipsnames]{xcolor}
\usepackage{pifont}
\usepackage{nicefrac}
\usepackage{sectsty}
\usepackage{lipsum}  

% ------
% Fonts and typesetting settings
\usepackage[sc]{mathpazo}
\usepackage[T1]{fontenc}
\linespread{1.1} % Palatino needs more space between lines
\usepackage{microtype}
\subsectionfont{\fontsize{10}{15}\selectfont}

% ------
% Page layout
\usepackage[hmarginratio=1:1,top=32mm,columnsep=20pt]{geometry}
\usepackage[font=it]{caption}
\usepackage{paralist}
\usepackage{multicol}

% ------
% Abstract
\usepackage{abstract}
	\renewcommand{\abstractnamefont}{\normalfont\bfseries}
	\renewcommand{\abstracttextfont}{\normalfont\small\itshape}


% ------
% Titling (section/subsection)
\usepackage{titlesec}
\renewcommand\thesection{\Roman{section}}
\titleformat{\section}[block]{\large\scshape\centering}{\thesection.}{1em}{}

% ------
% Clickable URLs (optional)
\usepackage{hyperref}

% ------
% Header/footer
\usepackage{fancyhdr}
	\pagestyle{fancy}
%	\fancyhead{}
	\fancyfoot[C]{Wissenschaftliches Seminar WS 2017/18 $\cdot$
          Software Engineering $\cdot$ Prof. Skornia}
	\fancyhead[C]{OTH Regensburg $\cdot$ Fakultät IM \newline \newline } 
	\fancyfoot[RO,LE]{\thepage}



% ------
% Maketitle metadata
\title{\vspace{-5mm}%
	\fontsize{20pt}{10pt}\selectfont
	\textbf{Adventures in Attacking Wind Farm}
	}	
\vspace{-5mm}\date{}
\author{
	\large
       \begin{minipage}[t]{0.33\linewidth}
         \begin{center}
           	\textsc{Butrint Dehari}\\[2mm]
                 \normalsize	Matr.nr: 3118661\\
                 \normalsize
                 \href{mailto:autor1@stud.oth-regensburg.de}
                 {butrint.dehari@st.oth-regensburg.de}      
         \end{center}
       \end{minipage}        
       \begin{minipage}[t]{0.33\linewidth}
         \begin{center}
           	\textsc{Celestin Mouangue}\\[2mm]
                 \normalsize	Matr.nr: 3080216\\
                 \normalsize
                 \href{mailto:autor2@stud.oth-regensburg.de}
                 {autor2@stud.oth-regensburg.de}      
         \end{center}
       \end{minipage}
       \begin{minipage}[t]{0.33\linewidth}
         \begin{center}
           	\textsc{Zhong XU}\\[2mm]
                 \normalsize	Matr.nr: 123456\\
                 \normalsize
                 \href{mailto:autor3@stud.oth-regensburg.de}
                 {autor3@stud.oth-regensburg.de}      
         \end{center}
       \end{minipage}
     }




%%%%%%%%%%%%%%%%%%%%%%%%
\begin{document}

\tableofcontents

\maketitle
\thispagestyle{fancy}

\begin{abstract}
\noindent Here comes the Abstract...
\end{abstract}
	

\begin{multicols}{2}


\section{Chapter 1 : General Introduction}

\cite{author = {Trevor M. Letcher}, title = {Wind Energy Engineering}, journal = {Derp}, year = {2015}, volume = {1}, pages = {567},}
 \subsection{Why Wind Energy?}
 \subsubsection{Climate Change}
 Today we are threatened by global warming and climate change and therefore the energy industry should try to find energy sources free of carbon dioxide pollution and one of the options is generating electricity from wind energy. Wind and solar energy produce only 4\% of the global supply of electricity while coal, the worst fossil fuel polluter, is still the main energy source for generating electricity. So, producing electricity from wind is where we should be working towards in order to avoid any other environmentally unfriendly material for producing electricity. Hopefully with mass production and bigger and more efficient wind turbines, wind energy and other renewable forms of energy will become cheaper and more convenient to use than fossil fuel and therefore become the main source of producing energy.
 \subsubsection{Advantages of Wind Energy}
 Using wind turbines for electricity generation has many advantages and they have been the main reason for their rapid development.
 \begin{description}
 	\item[$\bullet$]
 	 \textit{Provision for a clean source of energy.} Wind energy delivers electricity without producing carbon dioxide. The relatively small amount of GHG emissions is produced in the manufacture and transport of the turbines and blades.
 	\item[$\bullet$] \textit{Sustainability.} Energy is sent to the grid whenever the sun shines and the wind blows and this makes wind a sustainable source of energy and another reason to invest in wind farms.
 	\item[$\bullet$] \textit{Location.} Wind turbines can be erected almost anywhere and often they are not in competition with urban development or other land usage.
 	\item[$\bullet$]
 	\textit{Stability of cost of electricity.} Once the wind farm is in place the cost of electricity to customers should be stable.
 	\item[$\bullet$]
 	\textit{Cost effectiveness.} Due to mass production and improved design, the cost of producing electricity from wind has significantly decreased.
 	\item[$\bullet$]
 	\textit{Creation of jobs and local resources.} Thousands of workers are employed for the different processes like the manufacture process, transport of turbines or erection of turbines. Wind Energy projects can be of great help in developing local resources, labor, capital, and even materials.
 \end{description}
 
 \subsubsection{Challenges facing the Wind Energy} Making use of the power of the wind comes with some challenges.
 \begin{description}
 	\item[$\bullet$]
 	\textit{Sporadicity of wind.} The most important problem with producing electricity from wind is that wind is unpredictable. It may not be blowing when the electricity from the wind farm is needed.  
 	\item[$\bullet$]
 	\textit{Good sites are often in remote locations.} This means that the electricity produced onshore has to be transported from these remote locations along expensive high-voltage cable to reach the customers.
 	\item[$\bullet$]
 	\textit{Noise pollution.} The noise coming from the rotating wind turbine falls exponentially with distance from the tower, and at 500 m the sound level is less than 35 dB which is not very much when normal conversation is rated at 60 dB.
 	\item[$\bullet$]
 	\textit{Turbine blades can damage wildlife.} The turning blades of wind turbines are the cause of more than 33, 000 bird deaths in the United States.
 	\item[$\bullet$]
 	\textit{Safety.} One of the main reasons that wind turbines are erected away from human habitation is because of safety issues. If a blade would come adrift it would cause serious harm to people or animals nearby.
 	\item[$\bullet$]
 	\textit{New and unfamiliar technology.} Most of the general engineers are unfamiliar with the wind turbines and therefore when installing new wind turbines in some rural areas, it is a good practice to have some trained staff around the area, in case of a malfunction.
 	\item[$\bullet$]
 	\textit{Initial cost.} This is the most serious drawback of setting up a new wind farm and also the main reason why many governments offer subsidies.
 	
 	TODO: ADD REFERENCE : Wind Energy Engineering, by Trevor M. Letcher.
 \end{description}

\subsection{Anatomy of a Wind Turbine}
 \subsubsection{Rotors and Blades}
 Und noch etwas Text... \cite{muster} \newline
 The rotor is the element that captures energy from the wind. The efficiency depends on the number of the blades in the rotor, their shape, their length and the speed at which the rotor turns. The modern blades have a similar shape to an airplane wing and therefore the shape creates lift as the wind passes over it. The blade length depends on the size of the turbine, on the specific blade design and on the site where it is going to generate power from the wind. Typically a 3 MW wind turbine might have a 45m blade at one site where the wind regime is very good and 55m blade to produce the same amount of energy at a different site. The tip speed ratio determines the speed at which a wind turbine rotor turns. It should be an optimum so that each blade should pass through the air and the turbulence it creates should have dissipated before the next blade arrives in the same position.
 TODO: ADD REFERENCE : Wind Power Generation by Paul Breeze.
\subsubsection{The Drive Train, Nacelles and Towers}
\begin{description}
\item[$\bullet$] 
The drive train connects the energy capturing rotor of a wind turbine to the generator which produces the unit's electrical power. The drive train is composed of the gearbox and the generator. The gearbox is responsible for connecting the low-speed shaft attached to the turbine blades to the high-speed shaft attached to the generator. The gearbox converts the slow rotation of the outer blades, typically 30-60 rpm, to the roughly 1000-1800 rpm that the generator needs to start producing energy.
\item[$\bullet$] 
The house of all the main components of the machine except the rotor is called a nacelle. The components include the drive shaft, a gearbox, the generator, a brake to stop the turbine rotating in very low or very high winds and a range of hydraulics and servo systems that control the blade pitch, the rotor speed and the orientation of the complete nacelle structure. Nacelles are usually equipped with a helicopter pad to make it easier for the maintenance crew to land on top of the nacelle. This is very useful on offshore where the maintenance can be difficult sometimes. 
\item[$\bullet$] 
The tower's purpose is to raise the rotor so that it's blades are clear off the ground and of any other obstacle. The higher the turbine rotor is mounted, the stronger the wind, so it is better to raise the nacelle on the highest tower possible. Usually a tower's height is two or three times the blade length. There are different types of towers including: tubular steel towers, lattice towers or concrete towers.

TODO: ADD REFERENCE : Wind Power Generation by Paul Breeze, Understanding Wind Power Technology: Theory, Deployment and Optimisation by Alois Schaffarczyk 
\end{description}

\section{Chapter 2 : Wind Farm Infrastructure }
 \subsection{Wind farm communication infrastructure}
  \subsubsection{Overview of IEC-61400-25 standard}
  	The IEC-61400-25 standard was developed because different Wind Power Plant(WPP) technologies from different manufacturers resulted in technical, implementation and maintenance difficulties. The focus of this standard is on the communication between power plant components like wind turbines and actors like SCADA systems. A SCADA (Supervisory control and data acquisition) system is a software system that is used for controlling, monitoring and analyzing a process. This software communicates with the controller that is running the actual process. Typically these controllers would be PLC(Programmable Logic Controllers) or RTU(Remote Control Unit). The SCADA system gathers real time information from these controllers and brings them to the system where they are presented in a Graphical User Interface to the operators that are running the process. The IEC-61400-25 standard is comprised of six parts, namely: 
  	\begin{description}
  		\item[$\bullet$] 
  			\textit{Part 1- Overall description of principles and models:} provides an overview of the standard, defines some of the terminology used in the following parts and outlines the underlying modeling concept.
  		\item[$\bullet$] 
  			\textit{Part 2 - Information models:} introduces and defines the data objects which are referred to as logical nodes, specific to a wind turbine communication.
  		\item[$\bullet$] 
  			\textit{Part 3 - Information exchange models:} describes the required mechanisms and protocols of data exchange, like authenticating a client, sending a control command or accessing the self-description of a device.
  		\item[$\bullet$] 
  			 \textit{Part 4 - Mapping to communication profiles:} defines the message format of the individual data exchange transactions. The already existing mappings are web services, IEC-61850-8-1 MMS, OPC XML DA, IEC 60870-5-104, and DNP3. 	
  		\item[$\bullet$]	
  		 	 \textit{Part 5 - Conformance testing:} specifies standardized procedures for verifying that a given implementation adheres to the standard.
  		\item[$\bullet$]	
  			\textit{Part 6 - Logical node classes and data classes for condition monitoring:} extends the defined name spaces with logical nodes and possibly new data classes for condition monitoring.
  			
  			TODO: ADD References-> https://ac.els-cdn.com/S1364032115011405/1-s2.0-S1364032115011405-main.pdf?_tid=583cbb5e-d90c-11e7-9a15-00000aab0f27&acdnat=1512403450_5759bf0568d848251e97ea1a7f6d3fef
  			
  			https://pdfs.semanticscholar.org/3522/1c7994907b073f1b2401c610792336a0df7b.pdf.
  			
  			http://ieeexplore.ieee.org/stamp/stamp.jsp?arnumber=4694219
  			
  	\end{description}
  \subsubsection{SOAP based web services}
  	NOTE: check the security of these services, research vulnerability issues about these services. So have in mind the security of these services besides the general point.
  	
  	check : http://ieeexplore.ieee.org/stamp/stamp.jsp?arnumber=4558422
  	http://ieeexplore.ieee.org/stamp/stamp.jsp?arnumber=991449
  \subsubsection{OPC XML-DA services}	
   	NOTE: check the security of these services, research vulnerability issues about these services. So have in mind the security of these services besides the general point.
   	
   	check: http://ieeexplore.ieee.org/stamp/stamp.jsp?arnumber=4020022
 \subsection{Wind farm power infrastructure}
  \lipsum[1]
\section{Chapter 3 : Security Breach and Ransom }

\subsection{Introduction}
 \textbf{Talk about the risk management in Security}
 \lipsum[1]


\subsection{Physical Attack}
 \subsubsection{Vandalism}
 Und noch etwas Text... \cite{muster} \newline
 \lipsum[1]
 \subsubsection{Physical Access}
 \lipsum[1]
 
\subsection{Network breach}
 \subsubsection{Windshark}
 \lipsum[1]
 \subsubsection{Windpoison}
 \lipsum[1]
 \subsubsection{Others}
 \textbf{Black Box}
 \lipsum[1]
 
\subsection{Ransom}
 \subsubsection{Bitcoin}
 \lipsum[1]
 \subsubsection{Operation}
 \lipsum[1]



\section{Chapter 4 : Attack Mitigation }

\subsection{Introduction}
 \textbf{Talk about the risk which we can prevent}
 \lipsum[1]



\subsection{Physical Access}
derpderp
 \subsubsection{Access Control}
 \lipsum[1]

 \subsubsection{Locking System}
 Und noch etwas Text... \cite{muster} \newline
 \lipsum[1]
 \subsubsection{Quick Response}
 \lipsum[1]

 
\subsection{Network Mitigation}
 \subsubsection{SSL/TLS}
 \lipsum[1]
 \subsubsection{VPN}
 \lipsum[1]
 \subsubsection{Access restriction}
 \textbf{White List}
 \lipsum[1]
 
\subsection{Best Practice}
 \lipsum[2]

 
 
\bibliographystyle{lit}
\bibliography{lit}
\end{multicols}

\end{document}



